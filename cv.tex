\documentclass[12pt,oldfontcommands]{memoir}
\usepackage{cv}

\title{Paul Kilgo}
\date{}

% Footer content
\cfoot{pkilgo@clemson.edu}
\lfoot{Paul Kilgo}
\rfoot{205.210.9536}

\begin{document}
\maketitle
\thispagestyle{fancy}

\section*{Academics}

\begin{itemize}[leftmargin=*]
  \item Ph.D. in Computer Science, Clemson University, 2012--present.
  \begin{itemize}
    \item Dissertation Topic:
          ``Numerical Computation of Feynman Path Integrals for Multiple
            Scattering Physics in Volume Rendering''
    \item Advisor: Jerry Tessendorf, Ph.D.
  \end{itemize}

  \item M.S. in Computer Science, The University of Alabama, 2012.
  \begin{itemize}
    \item Thesis: ``RDIS: A domain model for generalizing the mappings between
                    robotic software frameworks and robotic devices''
    \item Advisor: Monica Anderson, Ph.D.
  \end{itemize}

  \item B.S. in Computer Science and Mathematics, magna cum laude, The
        University of Alabama, 2011.
\end{itemize}

\section*{Research Experience}

Dissertation Research, Clemson University School of Computing, 2014--present.

\begin{itemize}
  \item Investigated methods for numerically computing Feynman path integrals
        for a formulation of the radiative transport equation.
  \item Benchmarked root-finding algorithms in a path generation and formulated
        an algorithm for asymptotic speedup.
\end{itemize}

Directed Projects, Clemson University School of Computing, Summer 2013.

\begin{itemize}
  \item Developed a node-based, rendering engine-agnostic plug-in for Blender
        for resolution independent volume modeling and grid computation using Python
        and SWIG.
\end{itemize}

Graduate Research Assistant, The University of Alabama Computer Science
Department, Summer 2012.

\begin{itemize}
  \item Developed and implemented a domain model for declarative specification
        of robotic device drivers.
  \item Introduced a visual modeling language for the domain model.
\end{itemize}

Research Assistant, Kilgo Headache Clinic, Summer 2008.

\begin{itemize}
  \item Wrote software using Swing and JDBC used by other research assistants
        for data entry.
  \item Performed data collection and entry for a retrospective study on
        patient retention in private medical practices.
\end{itemize}

Undergraduate Research Assistant, The University of Alabama Center for
Materials for Information Science, 2007.

\begin{itemize}
  \item Performed calibrations on gap separation for a vibrating sample
        magnetometer.
  \item Started an equipment documentation project using MediaWiki.
\end{itemize}

\section*{Teaching Experience}

Graduate Teaching Assistant, Clemson University School of Computing, 2012.

\begin{itemize}
  \item Co-led two lab sections (about 50 students) of CP SC 215: Software
        Development Foundations (Java). Graded projects for three sections (about 75
        students).
\end{itemize}

Graduate Teaching Assistant, The University of Alabama Computer Science
Department, 2011--2012.

\begin{itemize}
  \item Led a recitation session for one section of CS 360: Data Structures and
        Algorithm Analysis (about 30 students).
  \item Graded homework assignments for CS 470 and CS 570: Introduction to
        Computer Algorithms (about 50 students).
  \item Lectured and graded for one section (about 40 students) of CS 250:
        Programming II course (Python).
\end{itemize}

\section*{Professional Experience}

Graduate Lab Assistant, Clemson University School of Computing, 2013--present.

\begin{itemize}
  \item Assisted Systems Programmers in the management of 100+ Linux desktops
        (open lab and research lab) and servers using Puppet and related
        technologies.
  \item Led development for a Django application for arbitrary resource
        management, including VCS repositories and databases (about 100
        registered users). Contributed to several underlying open-source
        libraries dealing heavily with version control systems and SSH key
        management.
  \item Contributed to Django application for Mercurial-based homework and lab
        submissions (1,500+ registered users, 35,000+ repositories).
  \item Built custom Debian packages, set up custom Apt repositories, and
        managed a Jenkins instance for local development.
\end{itemize}


Student Assistant, The University of Alabama Center for Materials for
Information Science, 2007--2011.

\begin{itemize}
  \item Initiated and managed several Linux-based services including a
        motion-activated surveillance system, a faculty and student web space
        sign up system (PHP LAMP with Perl system scripts), and PXE imaging of
        Windows desktops.
  \item Assisted IT Manager in support of 80-100 Windows desktop and
        instrument-attached machines.
  \item Helped maintain two OSCAR Linux clusters.
\end{itemize}

\nocite{*}
\bibliographystyle{plainyr-rev}
\bibliography{cv}

\section*{Awards and Scholarships}

\begin{itemize}[leftmargin=*]
  \item Upsilon Pi Epsilon National Honor Society, Clemson University, 2013.
  \item ACM Outstanding Graduate Student, The University of Alabama, 2012.
  \item University Honors Program, The University of Alabama, 2007--2011.
  \item ACM Outstanding Undergraduate, The University of Alabama, 2010.
  \item Upsilon Pi Epsilon National Honor Society, The University of Alabama, 2009.
  \item Outstanding Junior in Computer Science, The University of Alabama, 2009.
\end{itemize}

\section*{Memberships and Service}

\begin{itemize}[leftmargin=*]
  \item Clemson Graduate School Professional Enrichment Grant, Reviewer, 2014.
  \item Journal of Software Engineering for Robotics, Reviewer, 2013.
  \item IEEE/RSJ International Conference on Intelligent Robots and
        Systems, Reviewer, 2012.
  \item The University of Alabama Student Chapter of the Association for
        Computing Machinery, Publicity Chair, 2011--2012.
  \item The University of Alabama Student Chapter of the Association for
        Computing Machinery, Member, 2009--2012.
  \item The Good Samaritan Clinic, Volunteer, Summer 2010.
\end{itemize}

\end{document}
